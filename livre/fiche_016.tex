
%%%%%%%%%%%%%%%%%% PREAMBULE %%%%%%%%%%%%%%%%%%

\documentclass[11pt,a4paper]{article}

\usepackage{amsfonts,amsmath,amssymb,amsthm}
\usepackage[utf8]{inputenc}
\usepackage[T1]{fontenc}
\usepackage[francais]{babel}
\usepackage{mathptmx}
\usepackage{fancybox}
\usepackage{graphicx}
\usepackage{ifthen}

\usepackage{tikz}   

\usepackage{hyperref}
\hypersetup{colorlinks=true, linkcolor=blue, urlcolor=blue,
pdftitle={Exo7 - Exercices de mathématiques}, pdfauthor={Exo7}}

\usepackage{geometry}
\geometry{top=2cm, bottom=2cm, left=2cm, right=2cm}

%----- Ensembles : entiers, reels, complexes -----
\newcommand{\Nn}{\mathbb{N}} \newcommand{\N}{\mathbb{N}}
\newcommand{\Zz}{\mathbb{Z}} \newcommand{\Z}{\mathbb{Z}}
\newcommand{\Qq}{\mathbb{Q}} \newcommand{\Q}{\mathbb{Q}}
\newcommand{\Rr}{\mathbb{R}} \newcommand{\R}{\mathbb{R}}
\newcommand{\Cc}{\mathbb{C}} \newcommand{\C}{\mathbb{C}}
\newcommand{\Kk}{\mathbb{K}} \newcommand{\K}{\mathbb{K}}

%----- Modifications de symboles -----
\renewcommand{\epsilon}{\varepsilon}
\renewcommand{\Re}{\mathop{\mathrm{Re}}\nolimits}
\renewcommand{\Im}{\mathop{\mathrm{Im}}\nolimits}
\newcommand{\llbracket}{\left[\kern-0.15em\left[}
\newcommand{\rrbracket}{\right]\kern-0.15em\right]}
\renewcommand{\ge}{\geqslant} \renewcommand{\geq}{\geqslant}
\renewcommand{\le}{\leqslant} \renewcommand{\leq}{\leqslant}

%----- Fonctions usuelles -----
\newcommand{\ch}{\mathop{\mathrm{ch}}\nolimits}
\newcommand{\sh}{\mathop{\mathrm{sh}}\nolimits}
\renewcommand{\tanh}{\mathop{\mathrm{th}}\nolimits}
\newcommand{\cotan}{\mathop{\mathrm{cotan}}\nolimits}
\newcommand{\Arcsin}{\mathop{\mathrm{arcsin}}\nolimits}
\newcommand{\Arccos}{\mathop{\mathrm{arccos}}\nolimits}
\newcommand{\Arctan}{\mathop{\mathrm{arctan}}\nolimits}
\newcommand{\Argsh}{\mathop{\mathrm{argsh}}\nolimits}
\newcommand{\Argch}{\mathop{\mathrm{argch}}\nolimits}
\newcommand{\Argth}{\mathop{\mathrm{argth}}\nolimits}
\newcommand{\pgcd}{\mathop{\mathrm{pgcd}}\nolimits} 

%----- Structure des exercices ------

\newcommand{\exercice}[1]{\video{0}}
\newcommand{\finexercice}{}
\newcommand{\noindication}{}
\newcommand{\nocorrection}{}

\newcounter{exo}
\newcommand{\enonce}[2]{\refstepcounter{exo}\hypertarget{exo7:#1}{}\label{exo7:#1}{\bf Exercice \arabic{exo}}\ \  #2\vspace{1mm}\hrule\vspace{1mm}}

\newcommand{\finenonce}[1]{
\ifthenelse{\equal{\ref{ind7:#1}}{\ref{bidon}}\and\equal{\ref{cor7:#1}}{\ref{bidon}}}{}{\par{\footnotesize
\ifthenelse{\equal{\ref{ind7:#1}}{\ref{bidon}}}{}{\hyperlink{ind7:#1}{\texttt{Indication} $\blacktriangledown$}\qquad}
\ifthenelse{\equal{\ref{cor7:#1}}{\ref{bidon}}}{}{\hyperlink{cor7:#1}{\texttt{Correction} $\blacktriangledown$}}}}
\ifthenelse{\equal{\myvideo}{0}}{}{{\footnotesize\qquad\texttt{\href{http://www.youtube.com/watch?v=\myvideo}{Vidéo $\blacksquare$}}}}
\hfill{\scriptsize\texttt{[#1]}}\vspace{1mm}\hrule\vspace*{7mm}}

\newcommand{\indication}[1]{\hypertarget{ind7:#1}{}\label{ind7:#1}{\bf Indication pour \hyperlink{exo7:#1}{l'exercice \ref{exo7:#1} $\blacktriangle$}}\vspace{1mm}\hrule\vspace{1mm}}
\newcommand{\finindication}{\vspace{1mm}\hrule\vspace*{7mm}}
\newcommand{\correction}[1]{\hypertarget{cor7:#1}{}\label{cor7:#1}{\bf Correction de \hyperlink{exo7:#1}{l'exercice \ref{exo7:#1} $\blacktriangle$}}\vspace{1mm}\hrule\vspace{1mm}}
\newcommand{\fincorrection}{\vspace{1mm}\hrule\vspace*{7mm}}

\newcommand{\finenonces}{\newpage}
\newcommand{\finindications}{\newpage}


\newcommand{\fiche}[1]{} \newcommand{\finfiche}{}
%\newcommand{\titre}[1]{\centerline{\large \bf #1}}
\newcommand{\addcommand}[1]{}

% variable myvideo : 0 no video, otherwise youtube reference
\newcommand{\video}[1]{\def\myvideo{#1}}

%----- Presentation ------

\setlength{\parindent}{0cm}

\definecolor{myred}{rgb}{0.93,0.26,0}
\definecolor{myorange}{rgb}{0.97,0.58,0}
\definecolor{myyellow}{rgb}{1,0.86,0}

\newcommand{\LogoExoSept}[1]{  % input : echelle       %% NEW
{\usefont{U}{cmss}{bx}{n}
\begin{tikzpicture}[scale=0.1*#1,transform shape]
  \fill[color=myorange] (0,0)--(4,0)--(4,-4)--(0,-4)--cycle;
  \fill[color=myred] (0,0)--(0,3)--(-3,3)--(-3,0)--cycle;
  \fill[color=myyellow] (4,0)--(7,4)--(3,7)--(0,3)--cycle;
  \node[scale=5] at (3.5,3.5) {Exo7};
\end{tikzpicture}}
}


% titre
\newcommand{\titre}[1]{%
\vspace*{-8ex} \hfill \hspace*{1.5cm} \hypersetup{linkcolor=black, urlcolor=black} 
\href{http://exo7.emath.fr}{\LogoExoSept{3}} \newline 
\hypersetup{linkcolor=blue, urlcolor=blue}  {\Large \bf #1} \newline 
 \rule{12cm}{1mm} \vspace*{3ex}}

%----- Commandes supplementaires ------



\begin{document}

%%%%%%%%%%%%%%%%%% EXERCICES %%%%%%%%%%%%%%%%%%
\fiche{f00016, bodin, 2007/09/01} 

\titre{Equations différentielles}

\exercice{847, bodin, 1998/09/01}
\enonce{000847}{}
On se propose d'int\'egrer sur l'intervalle le plus grand possible contenu dans $]0,
\infty[$ l'\'equation diff\'erentielle :

$$
(E) \ \ \ \ \ \ \ y'(x)- \frac{y(x)}{x}-y(x)^2=-9x^2.
$$
\begin{enumerate}
    \item D\'eterminer $ a \in ]0, \infty[ $ tel que $y(x)=ax$ soit une solution
particuli\`ere
$y_0$ de $(E)$.
    \item  Montrer que le changement de fonction inconnue : $y(x) = y_0(x)-\frac{1}{z(x)}$
transforme l'\'equation (E) en l'\'equation diff\'erentielle
$$
(E_1) \ \ \ \ \ \ \ z'(x) + (6x+ \frac{1}{x})z(x) =1.
$$
    \item Int\'egrer $(E_1)$ sur $]0,\infty[$.
    \item Donner toutes les solutions de $(E)$ d\'efinies sur $]0,\infty[$.
\end{enumerate}
\finenonce{000847} 
\finexercice
\exercice{863, bodin, 1998/09/01}
\enonce{000863}{}
R\'esoudre l'\'equation suivante :
$$ y^{\prime\prime}- 3y^\prime + 2 y = e^x. $$
\finenonce{000863} 
\finexercice
\exercice{864, bodin, 1998/09/01}
\enonce{000864}{}
R\'esoudre l'\'equation suivante :
 $$ y^{\prime\prime}- y =-6\cos x + 2x\sin x. $$
\finenonce{000864} 
\finexercice
\exercice{865, bodin, 1998/09/01}
\enonce{000865}{}
R\'esoudre l'\'equation suivante :
$$4y^{\prime\prime}+4y^\prime + 5y = \sin x e^{-x/2} .$$
\finenonce{000865} 
\finexercice
\exercice{866, bodin, 1998/09/01}
\enonce{000866}{}
 On consid\`ere l'\'equation  :
$$ y^{\prime\prime} + 2y^\prime + 4y = xe^x \qquad (E) $$
\begin{enumerate}
    \item  R\'esoudre l'\'equation diff\'erentielle homog\`ene associ\'ee \`a $(E)$.
    \item  Trouver une solution particuli\`ere de $(E)$ (expliquer votre
d\'emarche), puis donner l'ensemble de toutes les solutions de
$(E)$.
    \item  D\'eterminer l'unique solution $h$ de $(E)$ v\'erifiant
$h(0)=1$ et $h(1)=0$.
    \item  Soit $f : ]0,\infty [ \longrightarrow \R$
une fonction deux fois d\'erivable sur $]0,\infty [ $ et qui v\'erifie :
$$ t^2f^{\prime \prime}(t) +3tf^\prime (t) + 4f(t) = t\log{t}. $$
    \begin{enumerate}
        \item  On pose $g(x)=f(e^x)$, v\'erifier que $g$ est solution de $(E)$.
        \item  En d\'eduire une expression de $f$.
    \end{enumerate}
\end{enumerate}

\finenonce{000866} 


\finexercice
\exercice{872, bodin, 1998/09/01}
\enonce{000872}{}
On consid\`ere l'\'equation diff\'erentielle suivante :
$$
(E.D.) \quad y''-4y'+4y = d(x),
$$
o\`u $d$ est une fonction qui sera pr\'ecis\'ee plus loin.
\begin{enumerate}
    \item R\'esoudre l'\'equation diff\'erentielle homog\`ene (ou sans second membre)
associ\'ee \`a $(E.D.)$.
    \item Trouver une solution particuli\`ere de $(E.D.)$ lorsque $d(x)=e^{-2x}$
et lorsque $d(x)=e^{2x}$ respectivement.
    \item Donner la forme g\'en\'erale des solutions de $(E.D)$ lorsque
$$d(x) = \frac{e^{-2x}+e^{2x}}{4}.$$
\end{enumerate}
\finenonce{000872} 
\finexercice
\exercice{880, gourio, 2001/09/01}
\enonce{000880}{}
 R\'{e}soudre : $y''(x)+2y^{\prime }(x)+y(x)=2x\cos
x\cosh x$.

\finenonce{000880} 
\finexercice
\exercice{881, gourio, 2001/09/01}
\enonce{000881}{}
 D\'{e}terminer les $f\in C^{2}({\Rr},{\Rr}) $
telles que :
$$\forall x\in \Rr,f''(x)+f(-x)=x\cos x. $$
\finenonce{000881} 
\finexercice
\exercice{884, gourio, 2001/09/01}
\enonce{000884}{}
 En posant $t=\arctan x,$ r\'{e}soudre :
$$y''(x)+\frac{2x}{1+x^{2}}y^{\prime }(x)+\frac{y(x)}{(1+x^{2})^{2}}=0. $$
\finenonce{000884} 
\finexercice
\exercice{885, gourio, 2001/09/01}
\enonce{000885}{}
 R\'{e}soudre par le changement de fonction
$z=\frac{y}{x} $ l'\'{e}quation diff\'{e}rentielle :
$${x}^{2}y''(x)-2xy^{\prime }(x)+(2-x^{2})y(x)=0. $$
\finenonce{000885} 
\finexercice

\finfiche

 \finenonces 



 \finindications 

\noindication
\noindication
\noindication
\noindication
\noindication
\noindication
\noindication
\noindication
\noindication
\noindication


\newpage

\correction{000847}

% Correction : Nicolas Bonnotte

Le but de l'exercice est de résoudre l'équation
\begin{equation}\tag{E}\label{eq}
y'(x)- \frac{y(x)}{x}-y(x)^2=-9x^2.
\end{equation}

\begin{enumerate}
    \item Trouvons $a \in ]0, \infty[ $ tel que $y_0(x)=ax$ soit une solution
particuli\`ere. Puisque
\[ y_0'(x) - \frac{y_0(x)}{x}-y_0(x)^2 = - a^2 x^2,\]
$y_0$ est solution si et seulement si $a = \pm 3$. On choisit $a=3$.

    \item Si $z$ est une fonction $\mathcal{C}^1$ ne s'annulant pas, on pose $y(x) = 3x-1/{z(x)}$. Alors $y$ est solution si et seulement si
\[ \frac{z'(x)}{z(x)^2} + \frac{1}{xz(x)} - \frac{1}{z(x)^2} + \frac{6x}{z(x)} = 0.\]
En multipliant par $z(x)^2$, on obtient que $y$ est solution de \eqref{eq} ssi $z$ vérifie
\begin{equation}\tag{E$_1$}\label{eq1}
z'(x) + \left(6x+ \frac{1}{x}\right)z(x) =1.
\end{equation}


    \item On résout \eqref{eq1} sur $]0,\infty[$. Une primitive de $x \mapsto 6x + 1/x$ est $x \mapsto 3x^2 + \ln(x)$, donc les solutions de l'équation homogène sont les $x \mapsto A \exp(-3x^2 - \ln(x))$. On cherche une solution particulière de \eqref{eq1} sous la forme $z_p(x) = \alpha(x) \exp(-3x^2-\ln(x))$ ; alors $z_p$ est solution si
 $\alpha'(x) \exp(-3x^2 - \ln(x)) = 1$, c'est-à-dire si $\alpha'(x) = x\exp(3x^2)$, par exemple si $\alpha(x) = \exp(3x^2)/6$. Les solutions de \eqref{eq1} sont donc les 
 \[ z(x) = \frac{1+A\exp(-3x^2)}{6x}, \qquad \text{ avec } A \in \R.\]


   \item  On va maintenant en déduire les solutions de \eqref{eq} d\'efinies sur $]0,\infty[$. 
    
    Soit $y$ une solution $\mathcal{C}^1$ définie sur $]0,\infty[$. On suppose dans un premier temps que $y(x) > 3x$ sur l'intervalle ouvert $I \subset ]0,\infty[$, pris aussi grand que possible.  Alors $y(x) = 3x - 1/{z_I(x)}$ pour une certaine fonction $z_I < 0$ qui est $\mathcal{C}^1$ sur $I$. D'après la question précédente, on a nécessairement $z_I(x) = [1+A_I\exp(-3x^2)]/6x$ pour une certaine constante $A_I \in \R$. Puisque $z_I < 0$, cela impose $A_I<0$, mais du coup $I \neq ]0,+\infty[$ car $1 > A_I\exp(-3x^2)$ si $x$ est assez grand.
  
  Dans tous les cas, il existe donc un intervalle ouvert $J$ tel que $y(x) < 3x$ sur $J$. On suppose encore que $J$ est aussi grand que possible. Sur $J$, $y(x) = 3x - 1/{z_J(x)}$ pour une certaine fonction $z_J > 0$ qui est $\mathcal{C}^1$ sur $J$. Encore d'après la question précédente, $z_J(x) = [1+A_J\exp(-3x^2)]/6x$ pour une certaine constante $A_J$.
 Puisque l'intervalle ouvert $J = ]a,b[$ a été supposé maximal, et puisque $y$ est supposée définie sur $]0,+\infty[$, si $a > 0$ on a $y(a) = 3a$ et de même si $b < \infty$, $y(b) = 3b$, car sinon par continuité de $y$ on aurait encore $y(x) < 3x$ sur $]a-\epsilon,b+\epsilon[$ pour un petit $\epsilon > 0$. Cela n'est possible respectivement que si $z_J(x) \rightarrow +\infty$ lorsque $x \rightarrow a$ ou $z_J(x) \rightarrow +\infty$ lorsque $x \rightarrow b$. Or on a dit que $z_J = [1+A_J\exp(-3x^2)]/6x$, cela n'est donc pas possible du tout (sauf précisément si respectivement $a =0$ et $b=0$).
  
  Donc soit $y(x) = 3x$ sur $]0,+\infty[$, soit $y(x) < 3x$ sur $]0,+\infty[$. Dans ce dernier cas, $z(x) = 1/(3x-y(x))$ est définie sur $]0,+\infty[$ et s'écrit $z(x) = [1+A\exp(-3x^2)]/6x$. Puisque $z > 0$, nécessairement $A \geq -1$. Donc si $y$ est solution, alors     
    \[ y(x) = 3x \qquad \text{ou} \qquad y(x) =  3x - \frac{6x}{1+A\exp(-3x^2)} \quad \text{avec } A \geq -1.\]
    
 Réciproquement, si $y$ est ainsi définie, alors $y$ est définie et $\mathcal{C}^1$ sur $]0,\infty[$, et on peut vérifier que c'est bien une solution.
\end{enumerate}
\fincorrection
\correction{000863}
 $ y^{\prime\prime}- 3y^\prime + 2 y = e^x $. Le polyn\^ome caract\'eristique est
$f(r)= (r-1)(r-2)$ et les solutions de l'\'equation homog\`ene
sont donc toutes les fonctions :
$$ y(x) = c_1 e^x +c_2e^{2x}  \hbox{ avec } c_1, c_2 \in \R. $$
On cherche une solution particuli\`ere de la forme $y_p(x)=
P(x)e^x$, on est dans la situation $(\imath\imath)$ la condition
$(*)$ sur $P$ est : $P^{\prime\prime} -P^\prime = 1$, et $P(x)=-x$
convient. Les solutions de l'\'equation sont donc les fonctions :
$$ y(x) = (c_1-x)e^x +c_2e^{2x} \hbox{ avec } c_1, c_2 \in \R. $$
\fincorrection
\correction{000864}
 $ y^{\prime\prime}- y =-6\cos x + 2x\sin x $. Ici $f(r) =(r-1)(r+1)$ et l'\'equation
homog\`ene a pour solutions :
$$ y(x) = c_1 e^x +c_2e^{-x}  \hbox{ avec } c_1, c_2 \in \R .$$
On remarque que la fonction $3\cos x$ v\'erifie l'\'equation :
$y^{\prime\prime}- y =-6\cos x $, il nous reste donc \`a chercher
une solution $y_1$ de l'\'equation $y^{\prime\prime}-y=2x\sin x$,
car $y_p(x)=3\cos x+y_1(x)$ sera une solution de l'\'equation
consid\'r\'ee. Pour cela, on remarque que $2x\sin x = \mathop{\mathrm{Im}}\nolimits
(2xe^{ix})$ et on utilise la m\'ethode d\'ecrite plus haut pour
trouver une solution $z_1$ de l'\'equation : $y^{\prime\prime}- y
=2xe^{ix}$. On cherche $z_1$ sous la forme $P(x)e^{ix}$ o\`u $P$
est un polyn\^ome de degr\'e 1 car $f(i)=-2\not =0$. On a
$f^\prime(i)= 2i$, la condition $(*)$ sur $P$ est donc : $
2iP^\prime(x)-2P(x) = 2x $ ce qui donne apr\`es identification
$P(x) = -x-i$. Alors $y_1(x)=\mathop{\mathrm{Im}}\nolimits((-x+i)e^{ix})=-x\sin x-\cos x$.
Les solutions  sont par cons\'equent les fonctions :
$$y(x) = c_1 e^x +c_2e^{-x}+2\cos x -x\sin x  \hbox{ avec } c_1, c_2 \in \R. $$
Autre m\'ethode pour trouver une solution de $ y^{\prime\prime}- y
= 2x\sin x $  : On la cherche de la forme $y_1(x) = A(x)\sin x +
B(x)\cos x $ o\`u $A,B$ sont des polyn\^omes de degr\'e 1 car $i$
n'est pas racine de l'\'equation caract\'eristique ({\sl danger} :
pour un second membre du type $Q(x)\sin (\beta x)e^{\alpha x}$ la
discussion porte sur $\alpha+i\beta$ et non sur $\alpha$ ou
$\beta$...). On calcule $y_1^\prime$, $y_1^{\prime\prime}$ et on
applique l'\'equation \'etudi\'ee \`a $y_1$ \ldots on obtient la
condition :
$$ (A^{\prime\prime}-A-2B^\prime)\sin x +(B^{\prime\prime}-B-2A^\prime)= 2x\sin x$$
qui sera r\'ealis\'ee si : $ \left\lbrace \begin{array}{c}
                    A^{\prime\prime}-A-2B^\prime = 2x \\
                    B^{\prime\prime}-B-2A^\prime = 0
                \end{array} \right. . $ \\
On \'ecrit : $A(x)= ax+b$ et $B(x)=cx+d$, apr\`es identification
on obtient : $a=d=-1$, $b=c=0$, ce qui d\'etermine $y_1$.
\fincorrection
\correction{000865}
 $4y^{\prime\prime}+4y^\prime + 5y = \sin x e^{-x/2} $. L'\'equation
caract\'eristique a 2 racines complexes $r_1 = -1/2+i$ et
$r_2=\overline{r_1}$ et les solutions de l'\'equation homog\`ene
sont :
$$ y(x) = e^{-x/2}(c_1\cos x +c_2\sin x) \hbox{ avec } c_1,c_2\in \R $$
On a $\sin x e^{-x/2} =\mathop{\mathrm{Im}}\nolimits(e^{(-1/2+i)x})$, on commence donc par
chercher une solution $z_p$ de l'\'equation  avec le nouveau
second membre $e^{(-1/2+i)x}$.Comme $-1/2+i$ est racine de
l'\'equation caract\'eristique, on cherchera $
z_p(x)=P(x)e^{(-1/2+i)x}$ avec $P$ de degr\'e 1. Par cons\'equent
la condition $(*)$ sur $P$ :
$$ 4P^{\prime\prime}+f^\prime(-1/2+i)P^\prime+f(-1/2+i)P = 1$$
s'\'ecrit ici : $8iP^\prime =1$ ( $P^{\prime\prime} = 0$,
$f(-1/2+i)=0$ et $f^\prime(-1/2+i)=8i$), on peut donc prendre
$P(x)=-i/8x$ et $z_p(x)=-i/8xe^{(-1/2+i)x}$, par cons\'equent sa
partie imaginaire $y_p(x)=\mathop{\mathrm{Im}}\nolimits(-i/8xe^{(-1/2+i)x})= 1/8 x\sin
xe^{-x/2}$ est une solution de notre \'equation. Les solutions
sont donc toutes les fonctions de la forme :
$$y(x) = e^{-x/2}(c_1\cos x +(c_2+1/8x)\sin x) \hbox{ avec } c_1,c_2\in \R.$$

\fincorrection
\correction{000866}
\begin{enumerate}
    \item Le polyn\^ome caract\'eristique associ\'e \`a $E$ est : $p(x) = x^2+2x+4$ ; son discriminant
est $\Delta = -12$ et il a pour racines les 2 nombres complexes $
-1+i\sqrt{3}$ et $-1-i\sqrt3$. Les solutions de l'\'equation
homog\`ene sont donc toutes fonctions :
$$ y(x) = e^{-x}(a\cos{\sqrt3 x} + b\sin{\sqrt3 x})$$ obtenues lorsque $a,b$ d\'ecrivent $\R$.
    \item Le second membre est de la forme $ e^{\lambda x}Q(x) $ avec $\lambda = 1$ et $Q(x)=x$. On
cherchera une solution de l'\'equation sous la forme : $ y_p(x)=
R(x)e^x$ avec $R$ polyn\^ome de degr\'e \'egal \`a celui de $Q$
puisque $p(1) \not =0$. On pose donc $R(x) = ax+b$. On a
 $$ y_p^{\prime\prime}(x) + 2y_p^{\prime}(x) +4y_p(x) = (7ax+7b+4a)e^x.$$
  Donc $y_p$ est solution
si et seulement si $ 7ax + 7a+4b = x$. On trouve apr\`es
identification des coefficients :
 $$ a=\frac{1}{7} \qquad \hbox{et}\qquad b=\frac{-4}{49}.$$
La fonction $y_p(x)=\frac{1}{7}(x-\frac{4}{7})e^x$ est donc
solution de $E$ et la forme g\'en\'erale des solutions de $E$ est
:
$$ y(x)= e^{-x}(a\cos{\sqrt3 x} + b\sin{\sqrt3 x}) +\frac{1}{7}(x-\frac{4}{7})e^x \; ;\; a,b \in \R.$$

    \item  Soit $h$ une solution de $E$. Les conditions $h(0)=1$, $h(1)=0$ sont r\'ealis\'ees ssi
$$a=\frac{53}{49}\qquad \hbox{et} \qquad b=-\frac{53\cos\sqrt3+3e^2}{49\sin\sqrt3}.$$
    \item
    \begin{enumerate}
        \item  On a : $g^\prime(x) =e^xf^\prime(e^x)$ et
$g^{\prime\prime}(x)=e^xf^\prime(e^x)+e^{2x}f^{\prime\prime}(e^x)$
d'o\`u pour tout $x\in\R$ :
$$g^{\prime\prime}(x)+2g^\prime(x)+4g(x)=e^{2x}f^{\prime\prime}(e^x)+2e^xf^\prime(e^x)+4f(e^x)= e^x\log{e^x}=xe^x$$
donc $g$ est solution de $E$.
        \item  R\'eciproquement pour $f(t)= g(\log t)$ o\`u $g$ est une
solution de $E$ on montre que $f$ est 2 fois d\'erivable et
v\'erifie l'\'equation donn\'ee en 4. Donc les fonctions $f$
recherch\'ees sont de la forme :
$$\frac{1}{t}(a\cos{(\sqrt3\log t)} + b\sin{(\sqrt3\log t)}) +\frac{t}{7}(\log t-\frac{4}{7}) \; ;\; a,b \in \R.$$
    \end{enumerate}
\end{enumerate}
\fincorrection
\correction{000872}
\begin{enumerate}
    \item  L'\'equation caract\'eristique $r^2-4r+4=0$ a une racine (double) $r=2$
donc les solutions de l'\'equation homog\`ene sont les fonctions :
             $$ y(x) = (c_1x+c_2)e^{2x} \hbox{ o\`u } c_1,c_2 \in\R. $$
    \item Pour $d(x) = e^{-2x}$ on peut chercher une solution particuli\`ere de la forme : $ y_1(x)
= ae^{-2x} $ car $-2$ n'est pas racine de l'\'equation
caract\'eristique. On a $y_1'(x)= -2e^{-2x}$ et
$y_1''(x)=4ae^{-2x}$. Par cons\'equent $y_1$ est solution si et
seulement si :
             $$\forall x\in\R\quad  (4a -4(-2a)+4a)e^{-2x} = e^{-2x} $$
donc si et seulement si $ a =\frac{1}{16}$. \\ Pour $d(x) =e^{2x}$
on cherche une solution de la forme $y_2(x)=ax^2e^{2x}$, car $2$
est racine double de l'\'equation caract\'eristique. On a $y_2'(x)
= (2ax+2ax^2)e^{2x}$ et
$y_2''(x)=(2a+4ax+4ax+4ax^2)e^{2x}=(4ax^2+8ax+2a)e^{2x}$. Alors
$y_2$ est solution si et seulement si
$$\forall x\in\R\quad (4ax^2+8ax+2a-4(2ax+2ax^2)+4ax^2)e^{2x} =e^{2x} $$
donc si et seulement si $a=\frac{1}{2}$.
    \item  On d\'eduit du principe de superposition que la
fonction $$
y_p(x)=\frac{1}{4}(y_1(x)+y_2(x))=\frac{1}{64}e^{-2x}+\frac{1}{8}x^2e^{2x}$$
est solution de l'\'equation pour le second membre donn\'e dans
cette question, et la forme g\'en\'erale des solutions est alors :
$$ y(x)=(c_1x+c_2)e^{2x}+\frac{1}{64}e^{-2x}+\frac{1}{8}x^2e^{2x} \hbox{ o\`u }c_1,c_2\in\R.$$
\end{enumerate}
\fincorrection
\correction{000880}
R\'{e}ponse : $\left( \lambda x+\mu \right)
e^{-x}+\frac{e^{x}}{25}\left[ \left( 3x-4\right) \cos x-\left(
4x-2\right) \sin x\right] +\left( \sin x-x\cos x\right) e^{-x}$.
\fincorrection
\correction{000881}
R\'{e}ponse : $\frac{1}{2}\left( -x\cos x+\sin x\right) +\lambda
\cos x+\mu \sinh x.$
\fincorrection
\correction{000884}
R\'{e}ponse : $x\rightarrow \frac{\lambda x+\mu
}{\sqrt{1+x^{2}}},\left( \lambda ,\mu \right) \in {\Rr}^{2}.$
\fincorrection
\correction{000885}
R\'{e}ponse : $x\rightarrow \lambda x\sinh x+\mu x\cosh x,\left(
\lambda ,\mu \right) \in {\Rr}^{2}.$
\fincorrection


\end{document}

